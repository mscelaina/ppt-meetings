\documentclass{beamer}
\usepackage{ctex, hyperref}
\usepackage[T1]{fontenc}

% other packages
\usepackage{latexsym,amsmath,xcolor,multicol,booktabs,calligra}
\usepackage{graphicx,pstricks,listings,stackengine}

\author{userElaina}
\title{SNN}
\subtitle{}
\institute{人工智能学院}
\date{2023年08月17日}
\usepackage{JilinUniv}

% defs
\def\cmd#1{\texttt{\color{red}\footnotesize $\backslash$#1}}
\def\env#1{\texttt{\color{blue}\footnotesize #1}}
\definecolor{deepblue}{rgb}{0,0,0.5}
\definecolor{deepred}{rgb}{0.6,0,0}
\definecolor{deepgreen}{rgb}{0,0.5,0}
\definecolor{halfgray}{gray}{0.55}

\lstset{
    basicstyle=\ttfamily\small,
    keywordstyle=\bfseries\color{deepblue},
    emphstyle=\ttfamily\color{deepred},    % Custom highlighting style
    stringstyle=\color{deepgreen},
    numbers=left,
    numberstyle=\small\color{halfgray},
    rulesepcolor=\color{red!20!green!20!blue!20},
    frame=shadowbox,
}


\begin{document}

\kaishu
\begin{frame}
    \titlepage
    \begin{figure}[htpb]
        \begin{center}
            \includegraphics[width=0.15\linewidth]{pic/Jilin_University_Logo.eps}
        \end{center}
    \end{figure}
\end{frame}

\begin{frame}
    \tableofcontents[sectionstyle=show,subsectionstyle=show/shaded/hide,subsubsectionstyle=show/shaded/hide]
\end{frame}


\section{神经元模型}

\begin{frame}{IF 和 LIF 模型}
    \begin{itemize}
        \item\begin{equation*}
            \begin{aligned}
                \tau\frac{{\rm d}V(t)}{{\rm d}t} &= X(t) \\
                V_t &= V_{t-1}+\frac{1}{\tau}X_t
            \end{aligned}
        \end{equation*}
        \item\begin{equation*}
            \begin{aligned}
                \tau\frac{{\rm d}V(t)}{{\rm d}t} &= -V(t)+X(t) \\
                V_t &= (1-\frac{1}{\tau})V_{t-1}+\frac{1}{\tau} X_t
            \end{aligned}
        \end{equation*}
    \end{itemize}
\end{frame}

\section{结构}

\begin{frame}
    \begin{itemize}[<+-| alert@+>]
        \item 类似 ANN。
        \item 事件驱动。
    \end{itemize}
\end{frame}

\section{输入}

\begin{frame}{Rate coding}
    \begin{itemize}[<+-| alert@+>] % 当然,除了alert,手动在里面插 \pause 也行
        \item 随着刺激强度的增加,动作电位的频率也会相应增加。
        \item 归一化的。
        \item 黑:0\%;白:100\%。
        \item 时间长度:足够多的电峰可以对其平均数量进行有效的估算。
    \end{itemize}
\end{frame}

\begin{frame}{Temporal coding}
    \begin{itemize}[<+-| alert@+>]
        \item 传递的信息包含在电峰的发放时刻或者高频电峰的统计涨落中。
        \item 01010101 和 00110011。
    \end{itemize}
\end{frame}

\begin{frame}
    \begin{center}
        {\Huge\calligra Thanks!}
    \end{center}
\end{frame}

\end{document}